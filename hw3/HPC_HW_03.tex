% This is a template for doing homework assignments in LaTeX

\documentclass[12pt]{article} % This command is used to set the type of document you are working on such as an article, book, or presenation

\usepackage{geometry} % This package allows the editing of the page layout
\usepackage{amsmath}  % This package allows the use of a large range of mathematical formula, commands, and symbols
\usepackage{graphicx}  % This package allows the importing of images
\usepackage{enumitem} % for change symbols in enumeration.
\usepackage{array}
\usepackage[table]{xcolor}
\usepackage{longtable}
\usepackage{amssymb}

\newcommand{\question}[2][]{\begin{flushleft}
        \textbf{Question #1}: \textit{#2}

\end{flushleft}}
\newcommand{\sol}{\textbf{Solution}:} %Use if you want a boldface solution line
\newcommand{\maketitletwo}[2][]{\begin{center}
        \Large{\textbf{Assignment #1}
        
        High Performance Computing} % Name of course here
        \vspace{5pt}
        
        \normalsize{Chia-Hao Wu  % Your name here
        
        #2}        % Change to due date if preferred
        \vspace{15pt}
        
\end{center}}
\begin{document}
    \maketitletwo[3]{}  % Optional argument is assignment number
    %Keep a blank space between maketitletwo and \question[1]
    
    \question[2]{
      develop an efficient way to evaluate the function outside of the interval $x \in [-\frac{\pi}{4}, \frac{\pi}{4}]$ using symmetries.\vspace{10pt}\\
      \normalsize{The idea of implementation:}\vspace{5pt}\\
      The sin function, $sin(x)$, is a periodic function with period $2\pi$, and $sin(x) = -sin(x + \pi).$ Thus, we can find a larger enough power of Taylor series with accuracy to 12-digits in $[-\pi, \pi]$ and use the periodic characteristic of the function to evaluate the function outside of the interval $x \in [-\frac{\pi}{4}, \frac{\pi}{4}]$. To elaborate it, $\forall x \in \mathbb{R}$, $\exists n \in \mathbb{Z}$ s.t. $x - (2n + 1)\pi \in [-\pi, \pi]$. In addition, n can be calculated by $\lfloor\frac{x}{2\pi}\rfloor$. Then, $\forall x \in \mathbb{R}$, we can calculate $n$ such that $x - (2n + 1)\pi \in [-\pi, \pi]$. Finally,we can calculate the approximation of $sin(x - (2n + 1)\pi)$ by using the larger enough power of Taylor series and  $sin(x) = -sin(x - (2n + 1)\pi)$.
    }
    
    \question[3]{
      Parallel Scan in OpenMP.\vspace{5pt}\\
      \begin{enumerate}[label=\emph{(\arabic*)}]
        \item {
          The architecture you run it on
          \vspace{5pt}\\
          \begin{tabular}{|>{\centering\arraybackslash}p{3cm}|>{\centering\arraybackslash}p{0.6\textwidth}|}
            \hline
            \rowcolor{blue!90!yellow!20}Attribute Name & Value\\
            \hline
            Kernel name & Darwin\\
            \hline
            kernel-release & 19.3.0\\
            \hline
            kernel-version & Darwin Kernel Version 19.3.0: Thu Jan 9 20:58:23 PST 2020;\newline root:xnu-6153.81.5~1/RELEASE\_X86\_64\\
            \hline
            machine & x86\_64\\
            \hline
            processor & i386\\ [1ex]
            \hline
          \end{tabular}
        }
        \item {The number of cores of the processor\vspace{5pt}\\
        2 physical cores(One processor with 2 cores) and 4 logical cores.
        }
        \item {
          The time it takes\vspace{5pt}\\
          N = 100,000,000\vspace{5pt}\\
          Elapsed time of sequential-scan = 0.768196 s\vspace{5pt}\\
          \begin{longtable}{|>{\centering\arraybackslash}p{3cm}|>{\centering\arraybackslash}p{0.6\textwidth}|}
            \hline
            \rowcolor{blue!90!yellow!20}Thread numbers & Elapsed time of parallel-scan (s)\\
            \hline
            2 & 0.339048\\
            \hline
            3 & 0.28607\\
            \hline
            4 & 0.243918\\
            \hline
            5 & 0.254444\\
            \hline
            6 & 0.243156\\
            \hline
            7 & 0.249474\\ 
            \hline
            8 & 0.248885\\ 
            \hline
            9 & 0.233481\\ 
            \hline
            10 & 0.251564\\ 
            \hline
            100 & 0.264672\\ 
            \hline
            200 & 0.274078\\ 
            \hline
            300 & 0.303070\\ 
            \hline
            400 & 0.291980\\ 
            \hline
            500 & 0.294022\\ 
            \hline
            600 & 0.322702\\ 
            \hline
            700 & 0.328262\\ 
            \hline
            1000 & 0.342116\\ 
            \hline
            1500 & 0.396556\\ 
            \hline
            1000 & 0.489136\\ [1ex]
            \hline
          \end{longtable}
        }
      \end{enumerate}
    } 
    
\end{document}